\section{Synchronization and Deadlocks}

\paragraph{Producer-Consumer Problem}
\begin{itemize}
  \item \textbf{Definition}:
  \begin{itemize}
    \item buffer is shared between producer and consumer (LIFO)
    \item \code{count} integer keeps track of number of currently available items
    \item producer produces item $ \to $ placed in buffer, \code{count++}
    \item buffer full $ \to $ producer needs to sleep until consumer consumed an item
    \item consumer consumes item $ \to $ remove item from buffer, \code{count--}
    \item buffer empty $ \to $ consumer needs to sleep until producer produces item
  \end{itemize}
  \item \textbf{Problem}: \emph{race condition} on \code{count}
\end{itemize}

\paragraph{Producer-Consumer Problem --- condition variables}
\begin{itemize}
  \item \textbf{Solution}: can be solved with mutex + 2 counting semaphores
  \begin{itemize}
    \item hard to understand
    \item hard to get right
    \item hard to transfer to other problems
  \end{itemize}
  \item \textbf{condition variables}: allow blocking until condition is met
  \begin{itemize}
    \item usually suitable for same problems but much easier to get right
  \end{itemize}
  \item \textbf{Idea}:
  \begin{itemize}
    \item new operation performs \emph{unlock}, \emph{sleep}, \emph{lock} atomically
    \item new wake-up operation is called with lock held
    \item[$ \to $] simple mutex lock/unlock around CS + no signal loss
  \end{itemize}
  \item \textbf{Pthread} condition variables:
  \begin{itemize}
    \item \code{pthread_cond_init}: create + initialize new CV
    \item \code{pthread_cond_destroy}: destroy + free existing CV
    \item \code{pthread_cond_wait}: block waiting for signal
    \item \code{pthread_cond_timedwait}: block waiting for signal or timer
    \item \code{pthread_cond_signal} : signal another thread to wake up
    \item \code{pthread_cond_broadcast}: signal all threads to wake up
  \end{itemize}
\end{itemize}
\begin{figure}[h]\centering\label{ConditionVariables}\includegraphics[width=0.33\textwidth]{ConditionVariables}\end{figure}

\paragraph{Reader-Writer Problem}
\begin{itemize}
  \item \textbf{Problem}: model access to shared data structures
  \begin{itemize}
    \item many threads compete to read/write same data
    \item \emph{readers}: only read data set, not performing any updates
    \item \emph{writers}: both read and write
    \item[$ \leadsto $] using single mutex for read/write operations is not a good solution! (unnecessarily blocking out multiple readers while no writer is present)
  \end{itemize}
  \item \textbf{Idea}: locking should reflect different semantics for reading/writing
  \begin{itemize}
    \item no writing thread $ \to $ multiple readers may be present
    \item writing thread $ \to $ no other reader/writer allowed
  \end{itemize}
\end{itemize}

\paragraph{Dining-Philosophers Problem}
\begin{itemize}
  \item \textbf{Definition}: 5 philosophers with cyclic workflow:
  \begin{enumerate}
    \item think
    \item get hungry
    \item grab one chopstick
    \item grab other chopstick
    \item eat
    \item put down chopsticks
  \end{enumerate}
  \item \textbf{Rules}:
  \begin{itemize}
    \item no communication
    \item no atomic grabbing of both chopsticks
    \item no wrestling
  \end{itemize}
  \item \textbf{Abstraction}: models threads competing for limited number of resources
  \item \textbf{Problem}: what happens if all philosophers grab left chopstick at once?
  \item \textbf{Deadlock workarounds}:
  \begin{itemize}
    \item \emph{deadlock avoidance}: just 4 philosophers allowed at table of 5
    \item \emph{deadlock prevention}: odd philosophers take left chopstick first, even ones take right first
  \end{itemize}
\end{itemize}
\begin{figure}[h]\centering\label{DiningPhilosophers}\includegraphics[width=0.2\textwidth]{DiningPhilosophers}\end{figure}

\paragraph{Deadlocks}
\begin{itemize}
  \item \textbf{Deadlocks} can arise if all four conditions hold simultaneously:
  \begin{itemize}
    \item \emph{mutual exclusion}: limited resource access (can only be shared with finite number of users)
    \item \emph{hold and wait}: wait for next resource while already holding at least one
    \item \emph{no preemption}: granted resource cannot be taken away but only handed back voluntarily
    \item \emph{circular wait}: possibility of circularity in requests graph
  \end{itemize}
\end{itemize}

\paragraph{Deadlocks --- countermeasures}
\begin{itemize}
  \item \textbf{prevention}: pro-active, make deadlocks impossible to occur
  \item \textbf{avoidance}: decide on allowed actions based on a-priori knowledge
  \item \textbf{detection} (\emph{recovery}): react after deadlock happened
\end{itemize}

\paragraph{Deadlocks --- prevention}
\begin{itemize}
  \item \textbf{Goal}: negate at least one of the required deadlock conditions:
  \begin{itemize}
    \item \emph{mutual exclusion}: buy more resources, split into pieces, virtualize
    \item \emph{hold and wait}: get all resources en-bloque, 2-phase-locking
    \item \emph{no preemption}: virtualize to make preemptable
    \item \emph{circular wait}: reorder resources, prevent through partial order on resources
  \end{itemize}
\end{itemize}

\paragraph{Deadlocks --- avoidance}
\begin{itemize}
  \item \textbf{safe state}: system is in safe state $ \to $ there is a possible scheduling order in which all processes can finish without running into a deadlock
  \item \textbf{unsafe state}: system is in unsafe state $ \to $ a deadlock will definitely occur
  \item \textbf{avoidance}: on every resource request decide if system stays in safe state
  \begin{itemize}
    \item[$ \to $] \emph{resource allocation graph}
  \end{itemize}
\end{itemize}

\paragraph{Deadlock Avoidance --- resource allocation graph}
\begin{itemize}
  \item \textbf{principle}: view system state as graph
  \begin{itemize}
    \item \emph{processes} = round nodes
    \item \emph{resources} = square nodes
    \item \emph{resource instance} = dot in resource node
    \item \emph{resource requests/assignments} = edges
    \begin{itemize}
      \item resource $ \to $ process = resource is assigned to process
      \item process $ \to $ resource = process is requesting resource
    \end{itemize}
  \end{itemize}
  \begin{figure}[h]\centering\label{ResourceAllocationGraph}\includegraphics[width=0.12\textwidth]{ResourceAllocationGraph}\end{figure}
\end{itemize}

\paragraph{Deadlocks --- detection}
\begin{itemize}
  \item \textbf{Principle}: allow system to enter deadlock $ \to $ detection $ \to $ recovery scheme
  \item \textbf{wait-for graph} (WFG):
  \begin{itemize}
    \item \emph{processes} = nodes
    \item \emph{wait-for relationship} = edges
  \end{itemize}
  \item periodically invoke algorithm searching for cycle in graph
  \begin{itemize}
    \item[$ \leadsto $] cycle exists $ \to $ deadlock exists
  \end{itemize}
\end{itemize}
\begin{figure}[h]\centering\label{WaitForGraph}\includegraphics[width=0.33\textwidth]{WaitForGraph}\end{figure}

\paragraph{Deadlocks --- recovery}
\begin{itemize}
  \item \textbf{Process termination}:
  \begin{itemize}
    \item \emph{all}: abort all deadlocked processes
    \item \emph{selective}: abort one process at a time until deadlock is eliminated
  \end{itemize}
  \item \textbf{Termination order criteria}:
  \begin{itemize}
    \item process priority
    \item amount of computation time passed/left
    \item amount of resources used
    \item amount of resources needed for completion
    \item amount of processes to be terminated
    \item interactive or batch process
  \end{itemize}
  \item \textbf{Resource preemption}:
  \begin{itemize}
    \item \emph{victim selection}: minimize cost
    \item \emph{rollback}: perform periodic snapshots, abort process to preempt resources $ \to $ restart from last safe state
    \item \emph{starvation}: same process may always be picked as victim $ \leadsto $ include rollback count in cost factor
  \end{itemize}
\end{itemize}

\begin{summary}
  \begin{itemize}
    \item \textbf{classical synchronization problems}: model synchronization problems occurring in reality
    \begin{itemize}
      \item \emph{producer-consumer}: shared use of buffers/queues
      \item \emph{reader-writer}: shared access to data structures
      \item \emph{dining philosophers}: competition for limited resources
    \end{itemize}
    \item such synchronization problems occur very often when programming operating systems
    \item \textbf{parallelism}: introduced by multiple processors + multiprogramming, needs to be considered carefully when writing OS
  \end{itemize}
\end{summary}
