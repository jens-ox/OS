\section{Introduction to Operating Systems}

\textbf{What's an OS?}
\begin{items}
	\item \underline{abstraction}: provides abstraction for applications \\*
		manages and hides hardware details \\*
		uses low-level interfaces (not available to applications) \\*
		multiplexes hardware to multiple programs (\emph{virtualisation}) \\*
		makes hardware use efficient for applications
	\item \underline{protection}: \\*
		from processes using up all resources (\emph{accounting}, \emph{allocation}) \\*
		from processes writing into other processes memory
	\item \underline{resource managing}: \\*
		manages + multiplexes hardware resources \\*
		decides between conflicting requests for resource use \\*
		strives for efficient + fair resource use
	\item \underline{control}: \\*
		controls program execution \\*
		prevents errors and improper computer use
	\item \( \leadsto \) \textbf{no universially accepted definition}
\end{items}

\textbf{Hardware Overview}
\begin{items}
	\item CPU(s)/devices/memory (conceptually) connected to common bus
	\item CPU(s)/devices competing for memory cycles/bus
	\item all entities run concurrently
	\begin{figure}[H]\centering\label{BusSystem}\includegraphics[width=0.33\textwidth]{BusSystem}\end{figure}
	\item today: multiple busses
	\begin{figure}[H]\centering\label{BusSystemToday}\includegraphics[width=0.33\textwidth]{BusSystemToday}\end{figure}
\end{items}

\textbf{Central Processing Unit (CPU) -- Operation}
\begin{items}
	\item fetches instructions from memory, executes them (instruction format/-set depends on CPU)
	\item CPU internal registers store (meta-)data during execution (general purpose registers, floating point registers, instruction pointer (IP), stack pointer (SP), program status word (PSW),...)
	\item \underline{execution modes}: \\*
		\textbf{user mode} (x86: \emph{Ring 3}/\emph{CPL 3}): \\*
			\phantom{x} only non-privileged instructions may be executed \\*
			\phantom{x} cannot manage hardware \( \to \) \textbf{protection} \\*
		\textbf{kernel mode} (x86: \emph{Ring 0}/\emph{CPL 0}): \\*
			\phantom{x} all instructions allowed \\*
			\phantom{x} can manage hw with \textbf{privileged instructions}
		\begin{figure}[H]\centering\label{OperationModes}\includegraphics[width=0.2\textwidth]{OperationModes}\end{figure}
\end{items}

\textbf{Random Access Memory (RAM)}
\begin{items}
	\item keeps currently executed instructions + data
	\item today: CPUs have built-in \emph{memory controller}
	\item root complex connected directly via \\*
		"`wire"' to caches \\*
		pins to RAM \\*
		pins to PCI-E switches
\end{items}

\textbf{Caching}
\begin{items}
	\item RAM delivers instructions/data slower than CPU can execute
	\item memory references typicalle follow \emph{locality principle}: \\*
		\textbf{spatial locality}: future refs often near previous accesses \\*
			\phantom{x} (e.g. next byte in array) \\*
		\textbf{temporal locality}: future refs often at previously accessed ref \\*
			\phantom{x} (e.g. loop counter)
	\item \emph{caching} helps mitigating this memory wall: \\*
		copy used information temporarily from slower to faster storage \\*
		check faster storage first before going down \textbf{memory hierarchy} \\*
		if not, data is copied to cache and used from there
	\item \underline{Access latency}: \\*
		register: \( \sim \)1 CPU cycle \\*
		L1 cache (per core): \( \sim \)4 CPU cycles \\*
		L2 cache (per core pair): \( \sim \)12 CPU cycles \\*
		L3 cache/LLC (per uncore): \( \sim \)28 CPU cycles (\( \sim \)25 GiB/s) \\*
		DDR3-12800U RAM: \( \sim \)28 CPU cycles + \( \sim \) 50ns (\( \sim \)12 GiB/s)
\end{items}

\textbf{Caching -- Cache Organisation}
\begin{items}
	\item caches managed in hardware
	\item divided into \emph{cache lines} (usually 64 bytes each, unit at which data is exchanged between hierarchy levels)
	\item often separation of data/instructions in faster caches (e.g. L1, see \emph{harward architecture})
	\item \textbf{cache hit}: accessed data already in cache (e.g. L2 cache hit)
	\item \textbf{cache miss}: accessed data has to be fetched from lower level
	\item cache miss types: \\*
		\emph{compulsory miss}: first ref miss, data never been accessed \\*
		\emph{capacity miss}: cache not large enough for process working set \\*
		\emph{conflict miss}: cache has still space, but collisions due to \\* \phantom{x} placement strategy
\end{items}